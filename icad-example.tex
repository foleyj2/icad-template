%%%%%%%%%%%%%%%%%%%%%%% file template.tex %%%%%%%%%%%%%%%%%%%%%%%%%
% $Id: icad-example.tex 163 2017-05-15 23:02:51Z foley $
% $URL: https://repository.cs.ru.is/svn/template/tvd/journal/matec-woc/icad-example.tex $
% 
% This is a template file for Web of Conferences Journal
%
% Copy it to a new file with a new name and use it as the basis
% for your article
%
% This template has been updated to match the Word Template's contents
% by Joseph T. Foley < foley AT RU dot IS >
%
%%%%%%%%%%%%%%%%%%%%%%%%%% EDP Science %%%%%%%%%%%%%%%%%%%%%%%%%%%%
%
%%%\documentclass[option]{webofc}
%%% "twocolumn" for typesetting an article in two columns format (default one column)
%
\documentclass[twocolumn]{webofc}
\usepackage[varg]{txfonts}   % Web of Conferences font
\usepackage{booktabs}
\usepackage{array} %% needed for advanced table manipulation
%% Column types from http://tex.stackexchange.com/questions/54069/table-with-text-wrapping
\newcolumntype{L}[1]{>{\raggedright\let\newline\\\arraybackslash\hspace{0pt}}m{#1}}
\newcolumntype{C}[1]{>{\centering\let\newline\\\arraybackslash\hspace{0pt}}m{#1}}
\newcolumntype{R}[1]{>{\raggedleft\let\newline\\\arraybackslash\hspace{0pt}}m{#1}}

\graphicspath{{graphics/}{Graphics/}}  % where to look for graphics
%
% Put packages and macros that you use often in custom.sty
\usepackage{custom}

%% Fixmes are reminders of things that still need to be done.
%% The default fixmes are:  \fxnote{} \fxwarning{} \fxerror{} \fxfatal{} (same as \fixme{})
% if you want personalized fixmes, then register the authors here
% notice that the first field is 2 letters, the second is 3.
\FXRegisterAuthor{jf}{jtf}{foley}
% this registers \jfnote{}, \jfwarning{}, \jferror{}, \jffatal{}

\begin{document}
        %
        \title{International Conference of Axiomatic Design 2020 Template: MATEC Web of Conferences}
        %
        % subtitle is optionnal
        %
        %%%\subtitle{Do you have a subtitle?\\ If so, write it here}

        \author{\firstname{Joseph  T.} \lastname{Foley}\inst{1}\fnsep\thanks{\email{foley AT ru.is}} 
        \and
  \firstname{Miguel} \lastname{Cavique}\inst{2}
  % etc.
}

\institute{
  Reykjavik University, Menntavegur 1, Reykjavik 101, Iceland
  \and
  Naval Academy, Base Naval -- Alfeite, Almada, 2810--001, Portugal
}

\abstract{%
    Everyone who has written a conference paper as part of a team knows the challenges of integrating it all together while following the conference template appropriately.
    Axiomatic Design indicates that the design of a suitable process requires it to be modular and robust, which are not the case with the commonly used solution of WYSIWYG word processors such as Microsoft Word.
    In this paper, we compare an alternative system called LaTeX in the context of modularity and robustness in comparison to Microsoft Word due to the structure of document creation.
    LaTeX gives the option to break a document into separate pieces including at-compile figures, tables, and additional material.
    LaTeX is also a more robust data format as is proven by selectively corrupting a set of input files until they no longer compile.
    Through the lens of Axiomatic Design, markup languages such as LaTeX are much more robust and modular than current WYSIWYG document editors used in the conference paper development. 
}
%
\maketitle
%
\section*{README!}\label{sec:readme}
\textbf{Template Author's Notes (2019--02--11):}
  This version of the template has descriptive sections and examples for an Axiomatic Design/Product Design paper.
  If you are looking for the original MATEC LaTeX template, it can be found in \path{woc_2col.tex}.
  The original template has a lot of detailed instructions on formatting the paper, but most of that is already taken care of if you are using LaTeX.

  This template takes care of formatting and placement of the citations, as long as you fill in the BibTeX file \path{references.bib} and \path{references-ad.bib} correctly.
  If you are using crossrefs (to fill in the conference proceeding information), they must go at the end of the \path{.bib} file.

        If you have \path{woc.bst} it will ensure that the citations meet the formatting required by MATEC Web of Science with some modifications needed for the International Conference on Axiomatic Design.
        If you need the old citation format, just copy \path{woc-orig.bst} over the \path{woc.bst}.
        Enjoy!


\listoffixmes{}

\section*{\LaTeX{} Hints}
\begin{itemize}
\item Put one sentence per line.
  This makes it easier to debug errors (which are by line) and to do grammar checking with \url{http://grammarly.com}.
\item Compile the document often and look for errors.
  If you find one, try commenting out the area to locate the source of the problem.
\item Watch out for \& and \%.  They have to have a left-slash in front of them.
\end{itemize}

\section{Introduction}
\jfnote{Don't forget to write the introduction.}
\jfwarning{Really, don't forget to write the introduction.}
What is the idea?  What is it called and why?
Who is the target customer?

\subsection{Customer Needs}
What would a customer need the item to do?  
Using Axiomatic Design theory, this is stated as a numbered list of Customer Needs(CN)~\cite{suh1990principles}.
The top level is \CN0.
This is often (but not always) decomposed into \CN1, \CN2, etc.
Here is an example of a top level:

\begin{quote} \textbf{\CN0} A transfer bin for whole salmon, compatible with the SureTrack grader, cheaper and less prone to cracking due to skewing.  
The bin should be adaptable to a pure transfer task and be able to discharge anywhere along its path without
accidental discharge.~\cite{gerhard2016suretrack}
\end{quote}


\section{Prior Art}
What exists that is similar?  How is yours better/distinctive?
Give at least two examples and quantify the differences (numeric values).
If you say something is cheaper, you need to give the costs for both items.

An example of a figure is the grey square in Fig.~\ref{fig:grey-square}.
\begin{figure}
  \centering
  \includegraphics[width=\columnwidth]{grey-square}
  \caption{Grey square.}\label{fig:grey-square}
\end{figure}



\subsection*{Sources}
You will want to cite all these similar concepts/products.
As an example of a citation, Carryer et al.~\cite{carryer2011IntroMechatronics} is the textbook for T-411-MECH Mechatronics 1.


\section{Design}
As previously mentioned, using Axiomatic Design Theory is a good way to develop your design.

Here is a brief synopsis from Omarsdóttir et al.\cite{omarsdottir2016chessmate}:
\begin{quotation}
  Rather, the focus was placed on developing comprehensive FR and DP lists, then evaluating the coupling between them.
  This coupling is symbolized in a design matrix, which is a Cartesian product of all FR and DP combinations~\cite{cochran2016msdd, benevides2012aed}.
Where there is an interaction between an FR and DP, this is denoted by a non-zero coefficient, or in the case of the value being unknown, simply a placeholder variable $X$.
Minor levels of coupling, often considered higher-order effects, are annotated with $x$ to show their lessened effect.
A diagonal matrix is ``uncoupled'' and satisfies the Independence Axiom: ``to maintain the independence of the functional requirements~(FRs)''~\cite{suh2001axiomatic}.
Such a design can be easily optimized by adjusting a particular FR or DPs without affecting others.
A diagonal matrix indicates a ``decoupled'' or ``path-dependent'' solution, which can still be optimized, but the ordering of parameter choice selection becomes important.
All other design matrices are ``coupled'' and may have a usable local solution but usually resist modification and optimization~\cite{suh2001axiomatic}.
Needless to say, the focus is on minimizing coupling wherever it may appear.

ADT's second axiom is ``minimize the information content of the design.''
Simply put, ensure that the design has the highest probability of meeting the stated FRs.
When systems are not able to meet FRs all of the time, this is denoted in ADT as ``complexity'' and is deeply explored in~\cite{suh2005complexity}.
As will become apparent in the next section, this axiom became integral to the design of the interaction between the robot and its chess pieces.
Finally, any factors to be considered that are not functional are categorized as ``Constraints.''
These are often resource-focused and affect all of the design decisions; they need to be revisited often especially when choosing between otherwise equivalent implementations.
\end{quotation}
The first axiom is often called the Independence Axiom, and the second, the Information Axiom.


From the Customer Needs, we build a list of Functional Requirements.

Again, we start with a top-level \FR0: ``Contain \SI{25}{\kilogram} of fish on SureTrack conveyor until release is triggered''
From this, a top-level Design Parameter \DP0: Gable-reinforced stainless-steel locking bin with bi-directional discharge
\cite{gerhard2016suretrack}.

We continue a ``zig-zag'' procedure to decompose and map the FRs to the DPs as shown in Table~\ref{tab:first_level-frdp}.

\begin{table}
  \center
  \caption{First level FR-DP mapping.~\cite{gerhard2016suretrack}}\label{tab:first_level-frdp}
  \begin{tabular}{lll} \toprule
    ID& Functional Requirement & Design Parameter \\ \midrule 
    1&Contain product&Main weldment\\
    2&Move product&Support system\\
    3&Discharge product &Discharge system\\
    \bottomrule
  \end{tabular}
\end{table}

From this mapping we develop a design matrix as shown in Equation~\ref{eq:top-design-matrix} from~\cite{gerhard2016suretrack}.

\begin{equation}\label{eq:top-design-matrix}
\begin{Bmatrix}
\FR{1}\\
\FR{2}\\
\FR{3}
\end{Bmatrix}=
\begin{bmatrix}
X &  0 & X\\
0 &  X & 0\\
0 &  0 & X
\end{bmatrix}
\begin{Bmatrix}
\DP{1}\\
\DP{2}\\
\DP{3}
\end{Bmatrix}
\end{equation}

This matrix is de-coupled i.e.\ path-dependent, meaning it can be optimized, but the order matters.

\section{Experiments}

\section{Results and Discussion}

\section{Conclusion}

\subsection{Future work}

\subsection{Summary}

%\bibliography{references}
\bibliography{references-ad}

\end{document}
%%%%%%%%%%%%%%%%%%%% TeXStudio Magic Comments %%%%%%%%%%%%%%%%%%%%%
%% These comments that start with "!TeX" modify the way TeXStudio works
%% For details see http://texstudio.sourceforge.net/manual/current/usermanual_en.html   Section 4.10
%%
%% What encoding is the file in?
% !TeX encoding = UTF-8
%% What language should it be spellchecked?
% !TeX spellcheck = en_US
%% What program should I compile this document with?
% !TeX program = pdflatex
%% Which program should be used for generating the bibliography?
% !TeX TXS-program:bibliography = txs:///bibtex
%% This also sets the bibliography program for TeXShop and TeXWorks
% !BIB program = bibtex

%%% Local Variables:
%%% mode: latex
%%% TeX-master: t
%%% End:

