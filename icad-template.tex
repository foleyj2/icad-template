%%%%%%%%%%%%%%%%%%%%%%% file icad-template.tex %%%%%%%%%%%%%%%%%%%%%%
%% This is a template file for Springer LNCS Proceedings for       %%
%% the International Conference of Axiomatic Design as of 2026     %%
%%                                                                 %% 
%% Please do not use \input{...} to include other tex files.       %%
%% Submit your LaTeX manuscript as one .tex document.              %%
%%                                                                 %%
%% All additional figures and files should be attached             %%
%% separately and not embedded in the \TeX\ document itself.       %%
%%                                                                 %%
%%%%%%%%%%%%%%%%%%%%%%%%%%%%%%%%%%%%%%%%%%%%%%%%%%%%%%%%%%%%%%%%%%%%% 
\documentclass[runningheads]{llncs}

\usepackage[T1]{fontenc}
% T1 fonts will be used to generate the final print and online PDFs,
% so please use T1 fonts in your manuscript whenever possible.
% Other font encondings may result in incorrect characters.
%
\usepackage{amsfonts}
\usepackage{amsmath}
\usepackage{amssymb}
\usepackage{array}%% Design Matrix formatting
%% math fonts and symbols
%
\usepackage{graphicx}
% Used for displaying a sample figure. If possible, figure files should
% be included in EPS format.
%
\usepackage{mathtools}
%% Package for doing math regarding lengths (need for \columnwidth)
%
\usepackage{siunitx}
\newcommand{\SIkg}[1]{\SI{#1}{\kilo\gram}}
\newcommand{\SIma}[1]{\SI{#1}{\milli\ampere}}
%% Units formatting according to SI rules
%% with convenience macros
%
\usepackage{subcaption}
%% Makes sub captions inside for grouped figures.  Keeps all
%% the reference labels correct and makes "sub captions"
% 
\usepackage{booktabs}
%% packages to make tables suitable for print
%%%%%%%%%%%%%%%%%% Axiomatic Design Notation %%%%%%%%%%%%%%%%%%%%%%%%
\newcommand\CA[1]{\ensuremath{\text{CA}_{#1}}}
\newcommand\CN[1]{\ensuremath{\text{CN}_{#1}}}%product style
\newcommand\FR[1]{\ensuremath{\text{FR}_{#1}}}
\newcommand\DP[1]{\ensuremath{\text{DP}_{#1}}}
\newcommand\PV[1]{\ensuremath{\text{PV}_{#1}}}
\newcommand\CON[1]{\ensuremath{\text{CON}_{#1}}}
%%%%%%%%%%%%%%%%%%%%%%%%%%%%%%%%%%%%%%%%%%%%%%%%%%%%%%%%%%%%%%%%%%%%%
%
% to typeset URLs, URIs, and DOIs
% If you use the hyperref package, please uncomment the following two lines
% to display URLs in blue roman font according to Springer's eBook style:
\usepackage{hyperref}
\usepackage{color}
\hypersetup{
  colorlinks,
  linkcolor={black},
  citecolor={black},
  urlcolor={blue},
  }
\urlstyle{rm}
\usepackage{xurl}
%% Break long URLs in a sane way across long lines
%% xurl and hyperref should usually be the last package loaded

\begin{document}
\title{International Conference of Axiomatic Design 2026+ Template}
%
\titlerunning{ICAD2026+ Template}
% If the paper title is too long for the running head, you can set
% an abbreviated paper title here

\author{Joseph Timothy Foley\inst{1,2}\orcidID{0000-0003-2515-1799}
  \and Second Author\inst{3}\orcidID{0000-1111-2222-3333}
  \and Third Author\inst{3}\orcidID{1111-2222-3333-4444}
  \and Fourth Author\inst{3}\orcidID{2222-3333-4444-5555}}
\authorrunning{Foley \& Author et al.}% abbreviated author list (for running head)
\institute{Reykjavík University, Menntavegur 1, Reykjavík 102, Iceland\\
  \email{foley AT ru.is}\\
  \url{http://www.ru.is}\and
  Massachusetts Institute of Technology, 77 Massachusetts Avenue,Cambridge, Massachusetts 02139, USA\\
  \email{foley AT mit.edu}\\
  \url{http://www.mit.edu}}

\maketitle              % typeset the title of the contribution

\begin{abstract}
The abstract should briefly summarize the contents of the paper in
150--250 words.
This template attempts to integrate the Springer Lecture Notes on Computer Science Template with structure and notation common to Axiomatic Design.
\keywords{Axiomatic Design \and LaTeX Template \and Demonstration}
% Notice that we separate keywords with "\and"
\end{abstract}

\section*{Important Information}\label{sec:important-info}
This ICAD template is a customized version used in the Springer Computer Science Proceedings distributed at \url{https://www.springer.com/gp/computer-science/lncs/conference-proceedings-guidelines}
It is highly suggested to look at \url{https://resource-cms.springernature.com/springer-cms/rest/v1/content/19242230/data/v17} before writing your article:  it will save you a lot of hassle later.
You will want to remove or comment out these sections before you begin adding your content.

  This template takes care of formatting and placement of the citations, as long as you fill in the BibTeX file \path{references.bib} and \path{references-ad.bib} correctly.
  Advanced: If you are using crossrefs (to fill in the conference proceeding information), they must go at the end of the \path{.bib} file;  you can see examples of this near the end of \path{references-ad.bib}
  Reference processing is done by  \path|bibtex| and the \path|splncs04.bst| file.

  Springer's submission system wants everything to be in the same directory.
  This means that you need to put the graphics and supplementary material in the directory with the main \verb|.tex| file or it will put blank spots where they would go.
  The first author recommends starting the figure names with FIG to make them easier to put into the submission \verb|.zip| upload without including spurious files.
  They also want alt-text names for each of the figures which go into a separate Word file.
  ``Please try to avoid rasterized images for line-art diagrams and schemas. Whenever
possible, use vector graphics instead''~\cite{authorXXXXwordtemplate}

The rest of this document is a demonstration of capabilities and instructions for students who are writing their first conference paper.

\subsection*{\LaTeX{} Hints}\label{sec:latex-hints}
\begin{itemize}
\item Rename this file to something unique with the year and topic like: \path{icad2025-sharklasers.tex} so you can find it more easily.
\item Put one sentence per line.
  This makes it easier to debug errors (which are by line) and to do grammar checking with \url{http://grammarly.com}.
\item Compile the document often and look for errors.
  If you find one, try commenting out the area to locate the source of the problem.
\item Watch out for \& and \%.  They have to have a left-slash in front of them.
\item Underscore ``\verb|_|'' is only usable in math as a subscript.
  Don't put it in normal text.
\item Unicode non-ASCII characters can sometimes cause strange font errors.
  Avoid them unless absolutely necessary.
  The different versions of dash -, --, ---, and quotation ``,",'', and characters like \TH{} are often problematic if you don't use the \LaTeX{} macro.
\end{itemize}

\section{Demonstration}\label{sec:demonstration}
This section is taken directly from the LNCS template supplied by Springer to demonstrate the general formatting.

\subsection{A Subsection Sample}
Please note that the first paragraph of a section or subsection is
not indented. The first paragraph that follows a table, figure,
equation etc. does not need an indent, either.

Subsequent paragraphs, however, are indented.

\subsubsection{Sample Heading (Third Level)} Only two levels of
headings should be numbered. Lower level headings remain unnumbered;
they are formatted as run-in headings.

\paragraph{Sample Heading (Fourth Level)}
The contribution should contain no more than four levels of
headings. Table~\ref{tab1} gives a summary of all heading levels.

\begin{table}
\caption{Table captions should be placed above the
tables.}\label{tab1}
\begin{tabular}{|l|l|l|}
\hline
Heading level &  Example & Font size and style\\
\hline
Title (centered) &  {\Large\bfseries Lecture Notes} & 14 point, bold\\
1st-level heading &  {\large\bfseries 1 Introduction} & 12 point, bold\\
2nd-level heading & {\bfseries 2.1 Printing Area} & 10 point, bold\\
3rd-level heading & {\bfseries Run-in Heading in Bold.} Text follows & 10 point, bold\\
4th-level heading & {\itshape Lowest Level Heading.} Text follows & 10 point, italic\\
\hline
\end{tabular}
\end{table}


\noindent Displayed equations are centered and set on a separate
line.
\begin{equation}
x + y = z
\end{equation}
Please try to avoid rasterized images for line-art diagrams and
schemas. Whenever possible, use vector graphics instead (see
Fig.~\ref{fig1}).

\begin{figure}
\includegraphics[width=\textwidth]{fig1.eps}
\caption{A figure caption is always placed below the illustration.
Please note that short captions are centered, while long ones are
justified by the macro package automatically.} \label{fig1}
\end{figure}

\begin{theorem}
This is a sample theorem. The run-in heading is set in bold, while
the following text appears in italics. Definitions, lemmas,
propositions, and corollaries are styled the same way.
\end{theorem}
%
% the environments 'definition', 'lemma', 'proposition', 'corollary',
% 'remark', and 'example' are defined in the LLNCS documentclass as well.
%
\begin{proof}
Proofs, examples, and remarks have the initial word in italics,
while the following text appears in normal font.
\end{proof}
For citations of references, we prefer the use of square brackets
and consecutive numbers. Citations using labels or the author/year
convention are also acceptable. The following bibliography provides
a sample reference list with entries for journal
articles~\cite{author2016first}, an LNCS chapter~\cite{author2016inproceedings}, a
book~\cite{author1999book}, proceedings without editors~\cite{author2010inproceedingsnoed},
and a homepage~\cite{lcnsXXXXmisc}. Multiple citations are grouped
\cite{author2016first,author2016inproceedings,author1999book},
\cite{author2016first,author1999book,author2016inproceedings,lcnsXXXXmisc}.

\section{Introduction}\label{sec:introduction}
The Introduction section expands on the background of the work (some overlap with the Abstract is acceptable).
The introduction should not include subheadings.

What is the idea?  What is it called and why?
Who is the target customer?

\subsection{Customer Needs}\label{sec:customer-needs}
What would a customer need the item to do?  
Using Axiomatic Design theory, this is stated as a numbered list of Customer Needs(CN)~\cite{suh1990principles}.
The top level is \CA0 (or \CN0)
This is often (but not always) decomposed into \CA1, \CA2, etc.
Here is an example of a top level: \textbf{\CA0} A transfer bin for whole salmon, compatible with the SureTrack grader, cheaper and less prone to cracking due to skewing.  
The bin should be adaptable to a pure transfer task and be able to discharge anywhere along its path without
accidental discharge.~\cite{gerhard2016suretrack}

\section{Prior Art}\label{sec:prior-art}
What exists that is similar?
How is yours better/distinctive?
Give at least two examples and quantify the differences (numeric values).
If you say something is cheaper, you need to give the costs for both items.

An example of a figure is the four Axiomatic Design domains in Fig.~\ref{fig:ad-domains}.
\begin{figure}
  \centering
  \includegraphics[width=0.9\columnwidth]{FIG-ad-domains-zig}
  \caption{General progression between the Axiomatic Design domains employing the zig-zag approach}\label{fig:ad-domains}
\end{figure}


\subsection{Sources}\label{sec:sources}
You will want to cite all these similar concepts/products.
As an example of a citation, Carryer et al.~\cite{carryer2011IntroMechatronics} is the textbook for T-411-MECH Mechatronics 1.


\section{Design}\label{sec:design}
As previously mentioned, using Axiomatic Design Theory is a systematic way to develop your design from concept to a prototype.

Here is a brief synopsis from Omarsdóttir et al.\cite{omarsdottir2016chessmate}:
\begin{quotation}
  Rather, the focus was placed on developing comprehensive FR and DP lists, then evaluating the coupling between them.
  This coupling is symbolized in a design matrix, which is a Cartesian product of all FR and DP combinations~\cite{cochran2016msdd,benevides2012aed}.
Where there is an interaction between an FR and DP, this is denoted by a non-zero coefficient, or in the case of the value being unknown, simply a placeholder variable $X$.
Minor levels of coupling, often considered higher-order effects, are annotated with $x$ to show their lessened effect.
A diagonal matrix is ``uncoupled'' and satisfies the Independence Axiom: ``to maintain the independence of the functional requirements~(FRs)''~\cite{suh2001axiomatic}.
Such a design can be easily optimized by adjusting a particular FR or DPs without affecting others.
A diagonal matrix indicates a ``decoupled'' or ``path-dependent'' solution, which can still be optimized, but the ordering of parameter choice selection becomes important.
All other design matrices are ``coupled'' and may have a usable local solution but usually resist modification and optimization~\cite{suh2001axiomatic}.
Needless to say, the focus is on minimizing coupling wherever it may appear.

ADT's second axiom is ``minimize the information content of the design.''
Simply put, ensure that the design has the highest probability of meeting the stated FRs.
When systems are not able to meet FRs all of the time, this is denoted in ADT as ``complexity'' and is deeply explored in~\cite{suh2005complexity}.
As will become apparent in the next section, this axiom became integral to the design of the interaction between the robot and its chess pieces.
Finally, any factors to be considered that are not functional are categorized as ``Constraints.''
These are often resource-focused and affect all of the design decisions; they need to be revisited often especially when choosing between otherwise equivalent implementations.
\end{quotation}
The first axiom is often called the Independence Axiom, and the second, the Information Axiom.

``Time-dependent complexity'' is the same as what Suh describes as ``Information'' in his previous literature~\cite{suh1990principles,suh2001axiomatic}.
The Information content ($I$) for a design implementing Functional Requirement \FR{x} is:
\begin{equation}
  \label{eq:info}
  C_{\FR{x}}= I_{\FR{x}} = \log_{2}\frac{1}{p_{x}} = - \log_{2}p_{x}
\end{equation}
where $p_{x}$ is the probability that the choice of \DP{x} meets that requirement.


From the Customer Needs, we build a list of Functional Requirements.

Again, we start with a top-level \FR0: ``Contain \SI{25}{\kilogram} of fish on SureTrack conveyor until release is triggered''
From this, a top-level Design Parameter \DP0: Gable-reinforced stainless-steel locking bin with bi-directional discharge
\cite{gerhard2016suretrack}.

We continue a ``zig-zag'' procedure to decompose and map the FRs to the DPs as shown in Table~\ref{tab:first_level-frdp}.

\begin{table}
  \center
  \caption{First level FR-DP mapping.~\cite{gerhard2016suretrack}}\label{tab:first_level-frdp}
  \begin{tabular}{lll} \toprule
    ID& Functional Requirement & Design Parameter \\ \midrule 
    1&Contain product&Main weldment\\
    2&Move product&Support system\\
    3&Discharge product &Discharge system\\
    \bottomrule
  \end{tabular}
\end{table}

From this mapping we develop a design matrix as shown in Equation~\ref{eq:top-design-matrix} from~\cite{gerhard2016suretrack}.

\begin{equation}\label{eq:top-design-matrix}
\begin{Bmatrix}
\FR{1}\\
\FR{2}\\
\FR{3}
\end{Bmatrix}=
\begin{bmatrix}
X &  0 & X\\
0 &  X & 0\\
0 &  0 & X
\end{bmatrix}
\begin{Bmatrix}
\DP{1}\\
\DP{2}\\
\DP{3}
\end{Bmatrix}
\end{equation}

This matrix is de-coupled i.e.\ path-dependent, meaning it can be optimized, but the order matters.

\section{Results/Experiments/Prototypes}\label{sec:rep}

\section{Discussion}\label{sec:results-discussion}

\section{Conclusion}\label{sec:conclusion}

\subsection{Future work}\label{sec:future-work}

\subsection{Summary}\label{sec:summary}

\begin{credits}
\subsubsection{\ackname} A bold run-in heading in small font size at the end of the paper is
used for general acknowledgments, for example: This study was funded
by X (grant number Y).

\subsubsection{\discintname}
It is now necessary to declare any competing interests or to specifically
state that the authors have no competing interests. Please place the
statement with a bold run-in heading in small font size beneath the
(optional) acknowledgments\footnote{If EquinOCS, our proceedings submission
system, is used, then the disclaimer can be provided directly in the system.},
for example: The authors have no competing interests to declare that are
relevant to the content of this article. Or: Author A has received research
grants from Company W. Author B has received a speaker honorarium from
Company X and owns stock in Company Y. Author C is a member of committee Z.
\end{credits}


%\section*{References}\label{sec:references}
\bibliographystyle{splncs04}
\bibliography{references, references-ad}

\end{document}



%%% Local Variables:
%%% mode: latex
%%% TeX-master: t
%%% End:
