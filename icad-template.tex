%%%%%%%%%%%%%%%%%%%%%%% file icad-template.tex %%%%%%%%%%%%%%%%%%%%%%
%% This is a template file for Springer Proceedings for            %%
%% the International Conference of Axiomatic Design as of 2025     %%
%%                                                                 %%
%% Please do not use \input{...} to include other tex files.       %%
%% Submit your LaTeX manuscript as one .tex document.              %%
%%                                                                 %%
%% All additional figures and files should be attached             %%
%% separately and not embedded in the \TeX\ document itself.       %%
%%                                                                 %%
%% See svproc for many more options to this template               %%
%%%%%%%%%%%%%%%%%%%%%%%%%%%%%%%%%%%%%%%%%%%%%%%%%%%%%%%%%%%%%%%%%%%%% 
\documentclass{svproc}

%%Allow Icelandic characters
\usepackage[icelandic,english]{babel}
\usepackage[T1]{fontenc}

%% math fonts and symbols
\usepackage{amsfonts}
\usepackage{amsmath}
\usepackage{amssymb}
\usepackage{array}%% Design Matrix formatting

%% Package for doing math regarding lengths (need for \columnwidth)
\usepackage{mathtools}

%% Units formatting according to SI rules
\usepackage{siunitx} 

%% Makes sub captions inside for grouped figures.  Keeps all
%% the reference labels correct and makes "sub captions"
\usepackage{subcaption}

%% Put packages you reuse often into custom.sty to be loaded here
\usepackage{custom}

%% packages to make tables look nicer
\usepackage{booktabs}
\usepackage{array} %% needed for advanced table manipulation

% to typeset URLs, URIs, and DOIs
\usepackage{xurl}%% This should usually be the last package loaded
\def\UrlFont{\rmfamily}

\begin{document}
\title{International Conference of Axiomatic Design 2025+ Template}

\titlerunning{ICAD2025+ Template}% abbreviated title (for running
% head) also used for the TOC unless \toctitle is used

\newcommand{\orcid}[1]{\ensuremath{^{[#1]}}}

\author{Joseph Timothy Foley\inst{1,2}\orcid{0000-0003-2515-1799}
  \and Second Author\inst{3}\orcid{0000-1111-2222-3333}
  \and Third Author\inst{3}\orcid{1111-2222-3333-4444}
  \and Fourth Author\inst{3}\orcid{2222-3333-4444-5555}}
%% The WordTemplate has the email at the end --foley
\authorrunning{Foley \& Author et al.}% abbreviated author list (for running head)
\tocauthor{Joseph T. Foley, Second Author, Third Author, and Fourth Author}
%%%% list of authors for the TOC
% (use if author list has to be modified)
\institute{Department of Engineering, Reykjavík University, Menntavegur 1, Reykjavík 102, Iceland,
  %\\ WWW home page:\texttt{http://www.ru.is/staff/foley} %%No URL in Word Template
  \and
  Department of Mechanical Engineering, Massachusetts Institute of Technology, 77 Massachusetts Avenue,Cambridge, Massachusetts 02139, USA
  \and
  Welsh School of Architecture, Cardiff University, Bute Building, King Edward VII Avenue, Cardiff, CF10 3NB, UK\\
\email{foley AT ru.is},}


\maketitle              % typeset the title of the contribution

\begin{abstract}
The abstract should summarize the contents of the paper
using at least 70 and at most 150 words. It will be set in 9-point
font size and be inset 1.0 cm from the right and left margins.
There will be two blank lines before and after the Abstract. \dots
% We would like to encourage you to list your keywords within
% the abstract section using the \keywords{...} command.
\keywords{Axiomatic Design, LaTeX Template, Demonstration}
\end{abstract}

\section*{Important Information}\label{sec:important-info}
This ICAD template is a customized version used in the Springer Proceedings distributed at \url{https://www.springer.com/in/authors-editors/conference-proceedings/conference-proceedings-guidelines}.
It is highly suggested that you download the zip file at \url{https://resource-cms.springernature.com/springer-cms/rest/v1/content/19338728/data/v1} and look at \path{authinst.pdf} before writing your article:  it will save you a lot of hassle later.
You will want to remove or comment out these sections before you begin adding your content.
Some changes have been made to be closer to the distributed Microsoft Word template at the same location.

  This template takes care of formatting and placement of the citations, as long as you fill in the BibTeX file \path{references.bib} and \path{references-ad.bib} correctly.
  Advanced: If you are using crossrefs (to fill in the conference proceeding information), they must go at the end of the \path{.bib} file;  you can see examples of this near the end of \path{references-ad.bib}
  References are taken care of by bibtex and the \path|splncs03_unsrt.bst| file.
  \begin{quotation}
    For citations of references, we prefer the use of square brackets and consecutive numbers.
    Citations using labels or the author/year convention are also acceptable.
    The following bibliography provides a sample reference list with entries for journal articles~\cite{author2016first}, an LNCS chapter~\cite{author2016inproceedings}, a book~\cite{author1999book}, proceedings without editors~\cite{author2010inproceedingsnoed}, as well as a URL~\cite{lcnsXXXXmisc}. {\em Text from:~\cite{authorXXXXwordtemplate}}
  \end{quotation} 

  Springer's submission system wants everything to be in the same directory.
  This means that you need to put the graphics and supplementary material in the directory with the main \verb|.tex| file or it will put blank spots where they would go.
  The first author recommends starting the figure names with FIG to make them easier to put into the submission \verb|.zip| upload without including spurious files.
  They also want alt-text names for each of the figures which go into a separate Word file.
  ``Please try to avoid rasterized images for line-art diagrams and schemas. Whenever
possible, use vector graphics instead''~\cite{authorXXXXwordtemplate}

  
\subsection*{\LaTeX{} Hints}\label{sec:latex-hints}
\begin{itemize}
\item Rename this file to something unique with the year and topic like: \path{icad2025-sharklasers.tex} so you can find it more easily.
\item Put one sentence per line.
  This makes it easier to debug errors (which are by line) and to do grammar checking with \url{http://grammarly.com}.
\item Compile the document often and look for errors.
  If you find one, try commenting out the area to locate the source of the problem.
\item Watch out for \& and \%.  They have to have a left-slash in front of them.
\item Underscore ``\verb|_|'' is only usable in math as a subscript.
  Don't put it in normal text.
\end{itemize}


\section{Introduction}\label{sec:introduction}
The Introduction section expands on the background of the work (some overlap with the Abstract is acceptable).
The introduction should not include subheadings.

What is the idea?  What is it called and why?
Who is the target customer?



\subsection{Customer Needs}\label{sec:customer-needs}
What would a customer need the item to do?  
Using Axiomatic Design theory, this is stated as a numbered list of Customer Needs(CN)~\cite{suh1990principles}.
The top level is \CN0.
This is often (but not always) decomposed into \CN1, \CN2, etc.
Here is an example of a top level:

\begin{quote} \textbf{\CN0} A transfer bin for whole salmon, compatible with the SureTrack grader, cheaper and less prone to cracking due to skewing.  
The bin should be adaptable to a pure transfer task and be able to discharge anywhere along its path without
accidental discharge.~\cite{gerhard2016suretrack}
\end{quote}


\section{Prior Art}\label{sec:prior-art}
What exists that is similar?  How is yours better/distinctive?
Give at least two examples and quantify the differences (numeric values).
If you say something is cheaper, you need to give the costs for both items.

An example of a figure is the four Axiomatic Design domains in Fig.~\ref{fig:ad-domains}.
\begin{figure}
  \centering
  \includegraphics[width=0.9\columnwidth]{FIG-ad-domains-zig}
  \caption{General progression between the Axiomatic Design domains employing the zig-zag approach}\label{fig:ad-domains}
\end{figure}


\subsection{Sources}\label{sec:sources}
You will want to cite all these similar concepts/products.
As an example of a citation, Carryer et al.~\cite{carryer2011IntroMechatronics} is the textbook for T-411-MECH Mechatronics 1.


\section{Design}\label{sec:design}
As previously mentioned, using Axiomatic Design Theory is a good way to develop your design.

Here is a brief synopsis from Omarsdóttir et al.\cite{omarsdottir2016chessmate}:
\begin{quotation}
  Rather, the focus was placed on developing comprehensive FR and DP lists, then evaluating the coupling between them.
  This coupling is symbolized in a design matrix, which is a Cartesian product of all FR and DP combinations~\cite{cochran2016msdd,benevides2012aed}.
Where there is an interaction between an FR and DP, this is denoted by a non-zero coefficient, or in the case of the value being unknown, simply a placeholder variable $X$.
Minor levels of coupling, often considered higher-order effects, are annotated with $x$ to show their lessened effect.
A diagonal matrix is ``uncoupled'' and satisfies the Independence Axiom: ``to maintain the independence of the functional requirements~(FRs)''~\cite{suh2001axiomatic}.
Such a design can be easily optimized by adjusting a particular FR or DPs without affecting others.
A diagonal matrix indicates a ``decoupled'' or ``path-dependent'' solution, which can still be optimized, but the ordering of parameter choice selection becomes important.
All other design matrices are ``coupled'' and may have a usable local solution but usually resist modification and optimization~\cite{suh2001axiomatic}.
Needless to say, the focus is on minimizing coupling wherever it may appear.

ADT's second axiom is ``minimize the information content of the design.''
Simply put, ensure that the design has the highest probability of meeting the stated FRs.
When systems are not able to meet FRs all of the time, this is denoted in ADT as ``complexity'' and is deeply explored in~\cite{suh2005complexity}.
As will become apparent in the next section, this axiom became integral to the design of the interaction between the robot and its chess pieces.
Finally, any factors to be considered that are not functional are categorized as ``Constraints.''
These are often resource-focused and affect all of the design decisions; they need to be revisited often especially when choosing between otherwise equivalent implementations.
\end{quotation}
The first axiom is often called the Independence Axiom, and the second, the Information Axiom.

From the Customer Needs, we build a list of Functional Requirements.

Again, we start with a top-level \FR0: ``Contain \SI{25}{\kilogram} of fish on SureTrack conveyor until release is triggered''
From this, a top-level Design Parameter \DP0: Gable-reinforced stainless-steel locking bin with bi-directional discharge
\cite{gerhard2016suretrack}.

There are two kinds of constraints to consider, input constraints and system constraints. 
Input constraints relate to guidance from the stakeholder usually such as \CON{1} "Construction completed in 12 weeks".
System constraints often relate to where the device will be used such as \CON{2} "Operate in the harshest Icelandic weather for at least 2 years".


We continue a ``zig-zag'' procedure to decompose and map the FRs to the DPs as shown in Table~\ref{tab:first_level-frdp}.

\begin{table}
  \center
  \caption{First level FR-DP mapping.~\cite{gerhard2016suretrack}}\label{tab:first_level-frdp}
  \begin{tabular}{lll} \toprule
    ID& Functional Requirement & Design Parameter \\ \midrule 
    1&Contain product&Main weldment\\
    2&Move product&Support system\\
    3&Discharge product &Discharge system\\
    \bottomrule
  \end{tabular}
\end{table}

From this mapping we develop a design matrix as shown in Equation~\ref{eq:top-design-matrix} from~\cite{gerhard2016suretrack}.

\begin{equation}\label{eq:top-design-matrix}
\begin{Bmatrix}
\FR{1}\\
\FR{2}\\
\FR{3}
\end{Bmatrix}=
\begin{bmatrix}
X &  0 & X\\
0 &  X & 0\\
0 &  0 & X
\end{bmatrix}
\begin{Bmatrix}
\DP{1}\\
\DP{2}\\
\DP{3}
\end{Bmatrix}
\end{equation}

This matrix is de-coupled i.e.\ path-dependent, meaning it can be optimized, but the order matters.

\section{Results/Experiments/Prototypes}\label{sec:rep}

\section{Discussion}\label{sec:results-discussion}

\section{Conclusion}\label{sec:conclusion}

\subsection{Future work}\label{sec:future-work}

\subsection{Summary}\label{sec:summary}

%\section*{References}\label{sec:references}
\bibliographystyle{bst/splncs03_unsrt}
\bibliography{references, references-ad}

\end{document}



%%% Local Variables:
%%% mode: latex
%%% TeX-master: t
%%% End:
