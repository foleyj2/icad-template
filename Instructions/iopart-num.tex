\documentclass[12pt]{iopart}

\bibliographystyle{iopart-num}
\usepackage{citesort}
%% \usepackage[square,sort&compress]{natbib}

\newcommand{\BibTeX}{Bib\TeX}
\newcommand{\REVTeX}{REV\TeX}

\sloppy 

\begin{document}
\nocite{*}

\title{The \texttt{iopart-num} \BibTeX{} style}

\noindent \qquad \\[-6pt] \qquad Version 2.0\\\qquad December 21, 2006

\author{M~A~Caprio}

\address{Center for Theoretical Physics, Sloane Physics Laboratory, 
Yale University, New Haven, Connecticut 06520-8120, USA}

\section{Introduction}

The \texttt{iopart-num} \BibTeX{} style is intended for use in
preparing manuscripts for Institute of Physics Publishing journals,
including Journal of Physics.  It provides numeric citation with
Harvard-like formatting, based upon the specification in ``How to
prepare and submit an article for publication in an IOP journal using
\LaTeXe'' by Graham Douglas (2005).

The \texttt{iopart-num} package is available on the Comprehensive
\TeX{} Archive Network (CTAN) as \texttt{/biblio/bibtex/contrib/iopart-num}.

\section{General instructions}

To use the \texttt{iopart-num} style, include the command
\verb+\bibliographystyle{iopart-num}+ in the
document preamble.  The reference section is then inserted into the
document with the command \verb+\bibliography{...}+, where the names
of the necessary \BibTeX{} database files should be listed between the
braces.  Further general instructions on using \BibTeX{} may be found
in the \BibTeX{} documentation.

The \texttt{iopart-num} style is compatible with, but does not
require, the \texttt{iopart} document class.  It is also compatible
with, but does not require, the \texttt{natbib} package.  For
documents prepared using the \texttt{iopart} class but
\textit{without} \texttt{natbib}, the section header 
for the references must be manually inserted, with the command
\verb+\section*{References}+, and use of the \texttt{citesort} package
is recommended for proper formatting of the references in the text.
For documents prepared \textit{with}
\texttt{natbib}, the section header for the references appears automatically, 
and use of the \texttt{citesort} package is not necessary.  The
\texttt{natbib} package should be loaded with the options \verb+square+ and \verb+sort&compress+
to insure proper formatting of the references in the text,
\textit{i.e.}, with \verb+\usepackage[square,sort&compress]{natbib}+.

\section{Special bibliographic data fields}

Under IOP style conventions, journal names should be set in italic
type.  However, for journals with multiple lettered sections, the IOP
convention is that the journal section letter should appear in roman
type after the main journal name, \textit{e.g.}, ``\textit{J.\
Phys.\/} A''.  Most existing \BibTeX{} styles do not make special
provision for lettered sections.  Therefore, typically, the section
letter is either included as part of the journal name
\begin{verbatim}
  journal = "J. Phys. A",
  volume = "38",
\end{verbatim}
or as part of the volume number
\begin{verbatim}
  journal = "J. Phys.",
  volume = "A38",
\end{verbatim}
in the \BibTeX{} database entry.  The \texttt{iopart-num} style
instead introduces a new optional field \verb+section+ which can be
used to specify a journal section letter.  This section letter is set in
roman type.  Moreover, if the section letter already appears in
\textit{any} of the usual locations in the database entry (at the end
of the journal name, before the volume number, or after the volume
number),
\texttt{iopart-num} will recognize it and suppress its printing.
Therefore, when you are creating the
\BibTeX{} database entry for an article in a lettered journal section,
you can still include the section letter in the \verb+journal+ or
\verb+volume+ fields, for use with other \BibTeX{} styles, 
without adversely affecting the formatting for IOP journals.  For
example, the entry for ref.~\cite{caprio2005:coherent} can be
generated with
\begin{verbatim}
  journal = "J. Phys. A",
  section = "A",
  volume = "38",
\end{verbatim}
or
\begin{verbatim}
  journal = "J. Phys.",
  section = "A",
  volume = "A38",
\end{verbatim}
or simply
\begin{verbatim}
  journal = "J. Phys.",
  section = "A",
  volume = "38",
\end{verbatim}
in the \BibTeX{} database entry.  Note that section names longer than a
single letter are also supported (\textit{e.g.},
``\textit{Phys. Rev.\/} ST Accel. Beams'').

Journal issue numbers are not customarily included in references to
journal articles under the IOP formatting conventions.
Therefore, the
\texttt{iopart-num} style ignores the
\verb+number+ field for articles.  
However, in some periodicals (such as popular magazines or certain
journal online supplements), pagination restarts from 1 with each
issue.  For such periodicals, the issue number is an essential part of
the bibliographic information needed to identify an article.  The
\texttt{iopart-num} style therefore supports an additional field
\verb+issue+ in the \BibTeX{} database entry, which can be used to 
enforce printing of the issue number.  If a value is specified for
\verb+issue+, this value is included included parenthetically after the volume
number in the reference, as in
ref.~\cite{zamfir2005:132te-beta-enam04}.

The \texttt{iopart-num} style supports several additional data fields
(\verb+collaboration+, \verb+eid+, \verb+eprint+, 
\verb+numpages+, and \verb+url+) introduced in
\REVTeX{}~4.


\section{Examples}

The entries in the reference list below provide examples of the
formatting of various types of references, of varying complexity,
including journal articles, books (individual, multivolume, or in a
series), articles in books, theses, and unpublished references.  The
\BibTeX{} database entries used to generate these examples can be
found in the file \texttt{iopart-num.bib}.
Refs.~\cite{ex1,ex2,ex3,ex4,ex5,ex6,ex7,ex8} are based upon example entries
from the IOP guidelines.

\section*{References}
\bibliography{iopart-num}

\end{document}


%%% Local Variables:
%%% mode: latex
%%% TeX-master: t
%%% End:
